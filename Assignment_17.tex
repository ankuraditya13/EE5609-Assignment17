\documentclass[journal,12pt,twocolumn]{IEEEtran}

\usepackage{setspace}
\usepackage{gensymb}

\singlespacing


\usepackage[cmex10]{amsmath}

\usepackage{amsthm}

\usepackage{mathrsfs}
\usepackage{txfonts}
\usepackage{stfloats}
\usepackage{bm}
\usepackage{cite}
\usepackage{cases}
\usepackage{subfig}

\usepackage{longtable}
\usepackage{multirow}

\usepackage{enumitem}
\usepackage{mathtools}
\usepackage{steinmetz}
\usepackage{tikz}
\usepackage{circuitikz}
\usepackage{verbatim}
\usepackage{tfrupee}
\usepackage[breaklinks=true]{hyperref}
\usepackage{graphicx}
\usepackage{tkz-euclide}

\usetikzlibrary{calc,math}
\usepackage{listings}
    \usepackage{color}                                            %%
    \usepackage{array}                                            %%
    \usepackage{longtable}                                        %%
    \usepackage{calc}                                             %%
    \usepackage{multirow}                                         %%
    \usepackage{hhline}                                           %%
    \usepackage{ifthen}                                           %%
    \usepackage{lscape}     
\usepackage{multicol}
\usepackage{chngcntr}

\DeclareMathOperator*{\Res}{Res}

\renewcommand\thesection{\arabic{section}}
\renewcommand\thesubsection{\thesection.\arabic{subsection}}
\renewcommand\thesubsubsection{\thesubsection.\arabic{subsubsection}}

\renewcommand\thesectiondis{\arabic{section}}
\renewcommand\thesubsectiondis{\thesectiondis.\arabic{subsection}}
\renewcommand\thesubsubsectiondis{\thesubsectiondis.\arabic{subsubsection}}


\hyphenation{op-tical net-works semi-conduc-tor}
\def\inputGnumericTable{}                                 %%

\lstset{
%language=C,
frame=single, 
breaklines=true,
columns=fullflexible
}
\begin{document}


\newtheorem{theorem}{Theorem}[section]
\newtheorem{problem}{Problem}
\newtheorem{proposition}{Proposition}[section]
\newtheorem{lemma}{Lemma}[section]
\newtheorem{corollary}[theorem]{Corollary}
\newtheorem{example}{Example}[section]
\newtheorem{definition}[problem]{Definition}

\newcommand{\BEQA}{\begin{eqnarray}}
\newcommand{\EEQA}{\end{eqnarray}}
\newcommand{\define}{\stackrel{\triangle}{=}}
\bibliographystyle{IEEEtran}
\providecommand{\mbf}{\mathbf}
\providecommand{\pr}[1]{\ensuremath{\Pr\left(#1\right)}}
\providecommand{\qfunc}[1]{\ensuremath{Q\left(#1\right)}}
\providecommand{\sbrak}[1]{\ensuremath{{}\left[#1\right]}}
\providecommand{\lsbrak}[1]{\ensuremath{{}\left[#1\right.}}
\providecommand{\rsbrak}[1]{\ensuremath{{}\left.#1\right]}}
\providecommand{\brak}[1]{\ensuremath{\left(#1\right)}}
\providecommand{\lbrak}[1]{\ensuremath{\left(#1\right.}}
\providecommand{\rbrak}[1]{\ensuremath{\left.#1\right)}}
\providecommand{\cbrak}[1]{\ensuremath{\left\{#1\right\}}}
\providecommand{\lcbrak}[1]{\ensuremath{\left\{#1\right.}}
\providecommand{\rcbrak}[1]{\ensuremath{\left.#1\right\}}}
\theoremstyle{remark}
\newtheorem{rem}{Remark}
\newcommand{\sgn}{\mathop{\mathrm{sgn}}}
\providecommand{\abs}[1]{\left\vert#1\right\vert}
\providecommand{\res}[1]{\Res\displaylimits_{#1}} 
\providecommand{\norm}[1]{\left\lVert#1\right\rVert}
%\providecommand{\norm}[1]{\lVert#1\rVert}
\providecommand{\mtx}[1]{\mathbf{#1}}
\providecommand{\mean}[1]{E\left[ #1 \right]}
\providecommand{\fourier}{\overset{\mathcal{F}}{ \rightleftharpoons}}
%\providecommand{\hilbert}{\overset{\mathcal{H}}{ \rightleftharpoons}}
\providecommand{\system}{\overset{\mathcal{H}}{ \longleftrightarrow}}
	%\newcommand{\solution}[2]{\textbf{Solution:}{#1}}
\newcommand{\solution}{\noindent \textbf{Solution: }}
\newcommand{\cosec}{\,\text{cosec}\,}
\providecommand{\dec}[2]{\ensuremath{\overset{#1}{\underset{#2}{\gtrless}}}}
\newcommand{\myvec}[1]{\ensuremath{\begin{pmatrix}#1\end{pmatrix}}}
\newcommand{\mydet}[1]{\ensuremath{\begin{vmatrix}#1\end{vmatrix}}}
\numberwithin{equation}{subsection}
\makeatletter
\@addtoreset{figure}{problem}
\makeatother
\let\StandardTheFigure\thefigure
\let\vec\mathbf
\renewcommand{\thefigure}{\theproblem}
\def\putbox#1#2#3{\makebox[0in][l]{\makebox[#1][l]{}\raisebox{\baselineskip}[0in][0in]{\raisebox{#2}[0in][0in]{#3}}}}
     \def\rightbox#1{\makebox[0in][r]{#1}}
     \def\centbox#1{\makebox[0in]{#1}}
     \def\topbox#1{\raisebox{-\baselineskip}[0in][0in]{#1}}
     \def\midbox#1{\raisebox{-0.5\baselineskip}[0in][0in]{#1}}
\vspace{3cm}
\title{Assignment-17}
\author{Ankur Aditya - EE20RESCH11010}
\maketitle
\newpage
\bigskip
\renewcommand{\thefigure}{\theenumi}
\renewcommand{\thetable}{\theenumi}

\begin{abstract}
This document contains the problem related to Eigenvalue and Eigenvectors (UGC-June-2017 Maths Q-78) 
\end{abstract}
Download the latex-file from 
\begin{lstlisting}
https://github.com/ankuraditya13/EE5609-Assignment17
\end{lstlisting}

\section{Problem}
Let T be the linear operator on $\vec{R}^3$ which is represented in the standard ordered basis by the matrix
\begin{align}
\myvec{-9&4&4\\-8&3&4\\-16&8&7}
\label{Q}
\end{align}
Prove that T is diagonalizable by exhibiting a basis for $\vec{R}^3$, each vector of which is a characteristic vector of T. 
\section{Theorem}
\subsection{\textbf{Theorem 1}}
Let $\vec{T}$ be a linear operator on a finite-dimensional space $\vec{V}$. Let $c_1, \cdots, c_k$ be the distinct characteristic values of $\vec{T}$ and let $\vec{W}_i$ be the null space of $\vec{T}-c_i\vec{I}$. The following are equivalent.
\begin{enumerate}
\item[(i)] $\vec{T}$ is diagonalizable.
\item[(ii)] The characteristic polynomial for $\vec{T}$ is,
\begin{align}
f = (x-c_1)^d_1 \cdots (x - c_k)^{d_k}
\end{align}
and $\dim W_i = d_i, i=1,\cdots ,k$
\item[(iii)] $\dim W_1+\cdots+\dim W_k = \dim V$ 
\end{enumerate}  
\section{Solution}
Now let,
\begin{align}
\vec{A} = \myvec{-9&4&4\\-8&3&4\\-16&8&7}
\label{A}
\end{align} 
Solving $\mydet{\lambda \vec{I} - \vec{A}} = 0$
\begin{align}
\mydet{\lambda \vec{I} - \vec{A}} = \mydet{\lambda + 9&-4&-4\\8&\lambda - 3&-4\\16&-8&\lambda - 7}\\
\xleftrightarrow[]{C_2\leftarrow C_2-C_3} \mydet{\lambda + 9&0&-4\\8&\lambda + 1&-4\\16&-\lambda - 1&\lambda - 7}
\end{align}
\begin{align}
\therefore \mydet{\lambda \vec{I} - \vec{A}} = (\lambda + 1)\mydet{\lambda + 9&0&-4\\8&1&-4\\16&-1&\lambda - 7}\\
\xleftrightarrow[]{R_3\leftarrow R_3+R_2} (\lambda + 1)\mydet{\lambda + 9&0&-4\\8&1&-4\\24&0&\lambda - 11}\\
\implies (\lambda + 1)\mydet{\lambda + 9 & -4\\24& \lambda - 11}\\
\implies \mydet{\lambda \vec{I} - \vec{A}} = (\lambda + 1)^{2}(\lambda - 3) = 0\\
\implies \lambda_1 = -1, \lambda_2 = -1, \lambda_3 =3
\label{lambda}
\end{align} 
Now at $\lambda_1$ and $\lambda_3$,
\begin{align}
\vec{A} + \vec{I} = \myvec{-8&4&4\\-8&4&4\\-16&8&8}\\
\vec{A} - 3\vec{I} = \myvec{-12&4&4\\-8&0&4\\-16&8&4}
\end{align}
Now we know that $\vec{A}-3\vec{I}$ is singular and rank($\vec{A}-3\vec{I})\geq 2.$ Therefore, rank($\vec{A}-3\vec{I}$) = 2. Hence from the theorem-1 (iii) it is evident that rank($\vec{A}+\vec{I}) = 1$. Let $X_1$ and $X_3$ be the spaces of characteristic vectors associated with the characteristic values 1 and 3 respectively. We know from rank nullity theorem that $\dim X_1 = 2$ and $\dim X_3 = 1$. Hence by Theorem-2 (i) $\vec{T}$ is diagonalizable. \\
The null-space of $\vec{T+I}$ is spanned by the vectors,
\begin{align}
\alpha_1 = \myvec{1&0&2}\\
\alpha_2 = \myvec{1&2&0}
\end{align}
As both $\alpha_1$ and $\alpha_2$ are independent, hence they form basis for $X_1$. The null-space of $\vec{T-3I}$ is spanned by the vector,
\begin{align}
\alpha_3 = \myvec{1&1&2}
\end{align}
Here $\alpha_3$ is a characteristic vector and a basis for $\vec{X_3}$.
Also the matrix P which enables us to change coordinates from the basis $\beta$ to the standard basis is the matrix which has transposes of $\alpha_1, \alpha_2 $ and $\alpha_3$ as it's column vectors:
\begin{align}
\vec{P} = \myvec{1&1&1\\0&2&1\\2&0&2}
\end{align} 
\end{document}